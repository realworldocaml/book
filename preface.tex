%
%
%
\chapter{Preface}

Objective Caml (OCaml) is a popular, expressive, high-performance dialect of ML developed
by a research team at INRIA in France.  This book presents a practical introduction and
guide to the language, with topics ranging from how to write a program to the concepts and
conventions that affect how affect how programs are developed in OCaml.  The text can be divided into
three main parts.

\begin{itemize}
\item The core language (Chapters \ref{chapter:expr1}--\ref{chapter:io}).
\item The module system (Chapters \ref{chapter:files}--\ref{chapter:functors}).
\item Objects and class (Chapters \ref{chapter:objects1}--\ref{chapter:polyclasses}).
\end{itemize}
%
This sequence is intended to follow the ordering of concepts needed as programs
grow in size (though objects and classes can be introduced at any point in the development).  It
also happens to follow the history of Caml: many of the core concepts were present in Caml and Caml
Light in the mid-1980s and early 1990s; Caml Special Light introduced modules in 1995; and Objective
Caml added objects and classes in 1996.

This book is intended for programmers, undergraduate and beginning graduate students
with some experience programming in a procedural programming language like C or Java, or
in some other functional programming language.  Some knowledge of basic data structures like lists,
stacks, and trees is assumed as well.

The exercises vary in difficulty.  They are intended to provide practice, as well as to investigate
language concepts in greater detail, and occasionally to introduce special topics not present
elsewhere in the text.

\paragraph{Acknowledgements}
This book grew out of set of notes I developed for Caltech CS134, an undergraduate course in
compiler construction that I started teaching in 2000.  My thanks first go to the many students who
have provided comments and feedback over the years.

My special thanks go to Tim Rentsch, who provided the suggestion and impetus for turning my course
notes into a textbook.  Tim provided careful reading and comments on earlier forms of the text.  In
our many discussions, he offered suggestions on topics ranging from programming language design and
terminology to writing style and punctuation.  Tim's precision and clarity of thought have been
indispensible, making a lasting impression on me about how to think and write about programming languages.

Heather Bergman at Cambridge University Press has been a most excellent editor, offering both
encouragement and help.
I also wish to thank Xavier Leroy and the members of the Caml team (project GALLIUM) at INRIA for
developing OCaml, and also for their help reviewing the text and offering suggestions.

Finally, I would like to thank Takako and Nobu for their endless support and
understanding.

% -*-
% Local Variables:
% Mode: LaTeX
% fill-column: 100
% TeX-master: "paper"
% TeX-command-default: "LaTeX/dvips Interactive"
% End:
% -*-
% vim:tw=100:fo=tcq:
