%
%
%
\exercises

\begin{exercise}{arith1}
For each of the following expressions, is the expression well-typed?  If it is well-typed, does it
evaluate to a value?  If so, what is the value?

\begin{enumerate}
\item

\lstinline!1 - 2!

\begin{answer}\ifanswers
Well typed.  The value is \hbox{\lstinline/-1/}.
\fi\end{answer}

\item

\lstinline!1 - 2 - 3!

\begin{answer}\ifanswers
Well typed.  Subtraction is left-associative, so the value is
\hbox{\lstinline/-4/}.
\fi\end{answer}

\item

\lstinline!1 - - 2!

\begin{answer}\ifanswers
Well typed.  The value is \hbox{\lstinline/3/}.
\fi\end{answer}

\item

\lstinline!0b101 + 0x10!

\begin{answer}\ifanswers
Well typed.  The value is \hbox{\lstinline/0x15/} (\hbox{\lstinline/21/} in decimal).
\fi\end{answer}

\item

\lstinline!1073741823 + 1!

\begin{answer}\ifanswers
Well typed.  On a 32-bit machine, \hbox{\lstinline/1073741823/} is the maximum
integer, so the value is \hbox{\lstinline/-1073741824/}.  On a 64-bit machine, the addition does not
overflow, so the result is \hbox{\lstinline/1073741824/}.
\fi\end{answer}

\item
\lstinline!1073741823.0 + 1e2!

\begin{answer}\ifanswers
Ill typed.  The operator \hbox{\lstinline/+/} is for integer addition
only.
\fi\end{answer}

\item
\lstinline!1 ^ 1!

\begin{answer}\ifanswers
Ill typed.  The operator \hbox{\lstinline/^/} is string concatenation.
\fi\end{answer}

\item

\lstinline!if true then 1!

\begin{answer}\ifanswers
Ill typed.  The missing \hbox{\lstinline/else/} branch has type
\hbox{\lstinline/unit/}, which is not compatible with \hbox{\lstinline/1/}.
\fi\end{answer}

\item

\lstinline!if false then ()!

\begin{answer}\ifanswers
Well typed.  The result is \hbox{\lstinline/()/}.
\fi\end{answer}

\item

\lstinline!if 0.3 -. 0.2 = 0.1 then 'a' else 'b'!

\begin{answer}\ifanswers
Well-typed.  On most machines, \hbox{\lstinline/0.3 -. 0.2/} is very close to, but different from,
  \hbox{\lstinline/0.1/}, so the result is \hbox{\lstinline/'b'/}.
\fi\end{answer}

\item

\lstinline!true || (1 / 0 >= 0)!

\begin{answer}\ifanswers
Well-typed.  The value is \hbox{\lstinline/true/} (since disjunction
\hbox{\lstinline/||/} is a short-circuit operator).
\fi\end{answer}

\item

\lstinline!1 > 2 - 1!

\begin{answer}\ifanswers
Well typed, because \hbox{\lstinline/-/} has higher precedence than \hbox{\lstinline/>/}.
The result is \hbox{\lstinline/false/}.
\fi\end{answer}

\item

\lstinline!"Hello world".[6]!

\begin{answer}\ifanswers
Well typed.  The value is \hbox{\lstinline/'w'/}.
\fi\end{answer}

\item

\lstinline!"Hello world".[11] <- 's'!

\begin{answer}\ifanswers
Well typed, but the index \hbox{\lstinline/11/} is out of bounds,
so the expression does not evaluate to a value.
\fi\end{answer}

\item

\lstinline!String.lowercase "A" < "B"!

\begin{answer}\ifanswers
Well typed.  The value is \hbox{\lstinline/false/}.
\fi\end{answer}

\item

\lstinline!Char.code 'a'!

\begin{answer}\ifanswers
Well typed.  The ASCII character code for \hbox{\lstinline/'a'/} is
\hbox{\lstinline/97/}.
\fi\end{answer}

\item \lstinline!(((())))!

\begin{answer}\ifanswers
Well typed.  The value is the unit \hbox{\lstinline/()/}.
\fi\end{answer}

\item \verb!((((*1*))))!

\begin{answer}\ifanswers
Well typed.  The value is \hbox{\lstinline/()/}.
\fi\end{answer}

\item \verb!((*((()*))!

\begin{answer}\ifanswers
Well typed.  The value is \hbox{\lstinline/()/}.
\fi\end{answer}
\end{enumerate}
\end{exercise}

% -*-
% Local Variables:
% Mode: LaTeX
% fill-column: 100
% TeX-master: "paper"
% TeX-command-default: "LaTeX/dvips Interactive"
% End:
% -*-
% vim:tw=100:fo=tcq:
